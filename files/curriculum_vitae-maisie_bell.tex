\documentclass[11pt,a4paper]{article}
\usepackage[letterpaper,margin=0.5in]{geometry}
\usepackage{mdwlist}
\usepackage{fontspec}
\usepackage{textcomp}
\usepackage{tgpagella}
\usepackage{titlesec}
\pagestyle{empty}
\setlength{\tabcolsep}{0em}
\setlength{\parindent}{0.5cm}
\usepackage{graphicx}
\DeclareGraphicsExtensions{.pdf,.png,.jpg}

\setmainfont{Apercu}
\setsansfont{Apercu-Mono}
\titleformat*{\subsection}{\sffamily\Large}
% indentsection style, used for sections that aren't already in lists
% that need indentation to the level of all text in the document
\newenvironment{indentsection}[1]%
{\begin{list}{}%
	{\setlength{\leftmargin}{#1}}%
	\item[]%
}
{\end{list}}

% opposite of above; bump a section back toward the left margin
\newenvironment{unindentsection}[1]%
{\begin{list}{}%
	{\setlength{\leftmargin}{-#1}}%
	\item[]%
}
{\end{list}}

% format two pieces of text, one left aligned and one right aligned
\newcommand{\headerrow}[2]
{\begin{tabular*}{\linewidth}{l@{\extracolsep{\fill}}r}
	#1 &
	#2 \\
\end{tabular*}}

% make "C++" look pretty when used in text by touching up the plus signs
\newcommand{\CPP}
{C\nolinebreak[4]\hspace{-.05em}\raisebox{.22ex}{\footnotesize\bf ++}}


% and the actual content starts here
\begin{document}

\begin{center}
% \includegraphics[scale=0.2]{./face2.jpg}\\
{\Huge \sffamily{Maisie Bell}}\\


% \ \ Student at UCL  \textbullet
% \ \ Flat 6 Holmdale Mansions\ \ \textbullet
% \ \ London NW6 1BG
% \ \ \faicon{github} mbellgb\ \ \textbullet\
% \ \ \faicon{medium} @mbell\_gb\ \ \textbullet\
% \ \ \faicon{twitter} @mbell\_gb\ \ \textbullet\
% \\
% +44 7827 638207\ \ \textbullet\
\ \ maisie@mbell.dev \textbullet\
\ \ https://mbell.dev/ \textbullet\
\ \ Twitter: @mbellgb
\end{center}


\hrule
\vspace{-1.2em}
\subsection*{Career}


\begin{unindentsection}{1.5em}
 \begin{itemize}
 	\item
 	\headerrow%
 		{\textbf{Apple Inc}}
		{\textbf{London, UK}}
 	\\
 	\headerrow%
 		{\emph{Site Reliability Engineer}}
 		{\emph{Jan 2021 --- present}}
	\begin{itemize*}
	    \item Site Reliability Engineer for the Object Storage team
            \item Working with host maintenance and repair automation at Apple scale
            \item On call for a highly critical service
            \item Working with a mix of bare metal and a wide variety of orchestration platforms
 	\end{itemize*}


 	\item
 	\headerrow%
 		{\textbf{Limejump Ltd}}
		{\textbf{London, UK}}
 	\\
 	\headerrow%
 		{\emph{Full Stack Developer}}
 		{\emph{Jun 2019 --- present}}
	\begin{itemize*}
	    \item Working on 10+ projects across the company, as part of the trading and operations team.
	    \item Leading and mentoring the team on migrating from ECS + Salt stack to Kubernetes, including running interactive workshops and demos.
	    \item Working multiple incident on-call rotas, getting exposure to the entire company's tech systems and fixing a wide array of issues.
	    \item Maintaining Django and AngularJS projects, as well as building features for new Golang \& Flask microservices.
	    \item Introduced new dev tools to the org: Kiali for monitoring service traffic and Jaeger for tracing requests from start to finish.
	    \item Added stronger integration testing for the trading platform by creating Kubernetes clusters and testing on them as part of the deployment pipeline.
 	\end{itemize*}

 	\item
 	\headerrow%
 		{\textbf{State Street}}
		{\textbf{London, UK}}
 	\\
 	\headerrow%
 		{\emph{Microservices Framework Engineer}}
 		{\emph{Jul 2018 --- May 2019}}
	\begin{itemize*}
	    \item Part of core engineering work to provide Kubernetes-as-a-service, to eventually be rolled out across the business
	    \item Contributing to high level design of a highly-available, resilient platform as well as implementation details
	    \item Rolled out and maintained developer tools (Concourse CI and Artifactory) for internal users
	    \item Working amongst constraints of corporate and industry regulations to provide a secure, compliant framework for developers
 	\end{itemize*}

 	\item
 	\headerrow%
 		{\textbf{Bell Software Development}}
		{\textbf{London, UK}}
 	\\
 	\headerrow%
 		{\emph{Various Contract Work}}
 		{\emph{Jul 2017 --Sep 2017}}
 	\begin{itemize*}
	    \item Various contracting work with a couple of startups during a summer break during my degree
	    \item Learned and used technologies like React, Firebase, and integrating data into a React Native app
 	\end{itemize*}
 \end{itemize}
\end{unindentsection}


 \vspace{-0.4em}
 \hrule
 \vspace{-1.2em}
 \subsection*{Education}

\begin{unindentsection}{1.5em}
\begin{itemize}
	\parskip=0.1em

	\item
	\headerrow%
		{\textbf{University College London}}
		{\textbf{2:1 awarded}}\\
	\headerrow%
		{\emph{BSc Computer Science}}
		{\emph{Sep 2015 --- Jul 2018}}

\end{itemize}
\end{unindentsection}

\vspace{-0.4em}
\hrule
\vspace{-1.2em}

\subsection*{Skills}

\begin{indentsection}{\parindent}
\hyphenpenalty=1000
\begin{description*}
	\item[Working experience with:]
	    Go, Python, Javascript \& Typescript, YAML
	\item[Web frameworks:]
	    Node.JS, React, and Django. Maintained Angular and Flask codebases.
	\item[Data and networking technologies:]
	    PostgreSQL, Nginx, UNIX shells and continuous integration systems (Concourse, Jenkins, CircleCI, GitHub Actions). Limited knowledge of Kafka.
	\item[DevOps:]
	    Experience building, running and maintaining Kubernetes
            services and cloud services, using provisioning tools like Ansible,
            Terraform and Salt, and using monitoring tools like Prometheus, Grafana
            and distributed tracing systems. Experience with Istio.
\end{description*}
\end{indentsection}

\vspace{-0.4em}
\hrule

\vspace{-1.2em}
\subsection*{Projects}
\begin{indentsection}{\parindent}
	More projects (plus source code) is available at https://mbell.me/projects.
	\begin{description*}
		\item[Healthcheck-Controller:] A Kubernetes controller to monitor service health (rather than pod health).
		\item[Haul:] A service for storing notes, thoughts, and other free-form information. Built with Ruby on Rails.
		\item[UCL Assistant:] An app to improve student life at university, using the UCL API to fetch timetables and study space availability.
	\end{description*}
\end{indentsection}

\end{document}
